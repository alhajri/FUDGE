\chapter{Joint energy-angle probability density tables}
\label{Sec:joint-table}
It is also possible to give energy-angle probability densities
as tables in \xendl.  These probability tables must be in the 
laboratory coordinate system.
The \ENDL~\cite{Omega} and \ENDF~\cite{ENDFB} forms of these
tables differ slightly, and \gettransfer\ supports both formats.
The \ENDF\ format is described first.

One format for tables
of values of $\pi(\Elab', \mulab \mid E)$ is as arrays
\begin{equation}
  \{ E, \{ \mulab, \{ \Elab', \pi(\Elab', \mulab \mid E) \} \} \}.
 \label{ENDF-E-mu-table}
\end{equation}
The data for the lowest incident energy $E$ are given first,
and data for a given incident energy are ordered by increasing direction cosine $\mulab$.
For fixed $E$ and $\mulab$, the data consist of pairs $\{ \Elab', \pi(\Elab', \mulab \mid E) \}$
for values of the energy $\Elab'$ of the outgoing particle.
The normalization of the data $\pi(\Elab', \mulab \mid E)$ is such that 
for each incident energy $E$
the total probability is
$$
  \int_0^\infty d\Elab' \, \int_{-1}^1 d\mulab \, \pi(\Elab', \mulab \mid E) = 1.
$$

The \ENDL\ energy-angle probability density data tables
are given in the form of the product
\begin{equation}
  \pi(\Elab', \mulab \mid E) =
  \pi_\mu(\mulab \mid E)\pi_E(\Elab' \mid E, \mulab),
   \label{correlated}
\end{equation}
in which $\pi_E(\Elab' \mid E, \mulab)$ is normalized so that
$$
  \int_0^\infty d\Elab' \, \pi_E(\Elab' \mid E, \mulab) = 1
$$
for each of the tabulated values of $E$ and~$\mulab$.

In the \gettransfer\ code energy-angle probability density
tables in the format of Eq.~(\ref{ENDF-E-mu-table}) are converted to
the format of Eq.~(\ref{correlated}) via the formulas
$$
  \pi_\mu(\mulab \mid E) = \int_0^\infty d\Elab' \, \pi(\Elab', \mulab \mid E)
$$
and
$$
  \pi_E(\Elab' \mid E, \mulab) = \frac
    {\pi(\Elab', \mulab \mid E)} {\pi_\mu(\mulab \mid E)}.
$$
The rest of the discussion of energy-angle probability density
tables is therefore in terms of the form of the data in Eq.~(\ref{correlated}).
Tbus, the discussion is in terms of the angular probability density $\pi_\mu(\mulab \mid E)$
and the outgoing energy conditional probability density $\pi_E(\Elab' \mid E, \mulab)$.

With the correlated energy-angle probability density
(\ref{correlated}) the number-preserving integral (\ref{Inum}) is
\begin{multline}
   \Inum_{gh,\ell} =
        \int_{\calE_g} dE \, \sigma ( E ) M(E) w(E) \widetilde \phi_\ell(E) 
       \, \int_{\calE_h' } d\Elab' \, \\
       \, \int_{\mulab} d\mulab \,  P_\ell( \mulab ) \pi_\mu(\mulab \mid E)\pi_E(\Elab' \mid E, \mulab),
  \label{Inum-corr}
\end{multline}
and the energy-preserving integral (\ref{Ien}) becomes
\begin{multline}
  \Ien_{gh,\ell} =
     \int_{\calE_g} dE \, \sigma ( E ) M(E) w(E) \widetilde \phi_\ell(E) 
     \, \int_{\calE_h' } d\Elab' \,  \Elab' \\
     \, \int_{\mulab} d\mulab  \,  P_\ell ( \mulab ) \pi_\mu(\mulab \mid E)\pi_E(\Elab' \mid E, \mulab).
  \label{Ien-corr}
\end{multline}

The method used by \gettransfer\ to evaluate the integrals (\ref{Inum-corr})
and (\ref{Ien-corr}) is to first compute the Legendre coefficients
\begin{equation}
  \pi_\ell(\Elab' \mid E) =
  \int_{-1}^1 d\mulab  \,  P_\ell ( \mulab ) \pi_\mu(\mulab \mid E)\pi_E(\Elab' \mid E, \mulab).
  \label{get-I4}
\end{equation}
The coding for the integration of (\ref{InumI4})
and~(\ref{IenI4}) is then applied to obtain the transfer matrix.

\section{Input of $\pi(\Elab', \mulab \mid E)$ the form of a table, Eq.~(\ref{ENDF-E-mu-table})}
For tables of the energy-angle probability density $\pi(\Elab', \mulab \mid E)$
in the format Eq.~(\ref{ENDF-E-mu-table}), the identification line in
Section~\ref{model-info} is\\
      \Input{Process: ENDF Double differential EMuEpP data}{}\\
These data are always in the laboratory frame,\\
  \Input{Product Frame: lab}{}

The first lines in the data for Section~\ref{model-info} give the number~$K$
of incident energies along with the interpolation rules\\
  \Input{EMuEpPData:}{$n = K$}\\
  \Input{Incident energy interpolation:}{probability interpolation flag}\\
  \Input{Outgoing cosine interpolation:}{probability interpolation flag}\\
  \Input{Outgoing energy interpolation:}{list interpolation flag}\\
The flags for interpolation with respect to incident energy~$E$ and
direction cosine~$\mulab$ are those for probability density tables in
Section~\ref{interp-flags-probability}, and that for outgoing energy~$E'$ is
one for simple lists.

For each incident energy~$E$ there is a data section of the form\\
    \Input{ Ein:}{$E$: \quad \texttt{n = $N$}}\\
indicating that data are given for $N$ values of $\mulab$.  The block of
data corresponding to a value of $\mulab$ is of the form\\
    \Input{ mu:}{$\mulab$: \quad \texttt{n = $J$}}\\
followed by $J$ pairs of values of outgoing energy $\Elab'$ and
probability density~$\pi(\Elab', \mulab \mid E)$ .

An example of such data with energy in MeV is\\
  \Input{EMuEpPData: n = 18}{}\\
  \Input{Incident energy interpolation: lin-lin unitbase}{}\\
  \Input{Outgoing cosine interpolation: lin-lin unibase}{}\\
  \Input{Outgoing energy interpolation: lin-lin }{}\\
  \Input{ Ein: 1.748830e+00:  n = 21}{}\\
  \Input{  mu: -1.000000e+00:  n = 15}{}\\
  \Input{ \indent   1.092990e-03  0.000000e+00}{}\\
  \Input{ \indent   1.093000e-03  7.406740e-01}{}\\
  \Input{ \indent   3.278900e-03  1.166140e+00}{}\\
  \Input{ \indent   7.650800e-03  1.466540e+00}{}\\
  \Input{ \indent   1.202300e-02  1.585880e+00}{}\\
  \Input{ \indent   2.076600e-02  1.610940e+00}{}\\
   \Input{ \indent 2.951000e-02  1.546240e+00}{}\\
\Input{ \indent  5.574100e-02  1.071950e+00}{}\\
\Input{ \indent  7.104300e-02  7.097100e-01}{}\\
\Input{ \indent  8.197300e-02  4.021720e-01}{}\\
 \Input{ \indent 9.071600e-02  1.795810e-01}{}\\
\Input{ \indent  9.508800e-02  9.526480e-02}{}\\
\Input{ \indent  9.946000e-02  2.867760e-02}{}\\
 \Input{ \indent 1.016500e-01  4.692750e-03}{}\\
\Input{ \indent  1.016510e-01  0.000000e+00}{}\\
\Input{ $\cdots$}{}\\
\Input{ Ein: 2.000000e+01:  n = 21}{}\\
 \Input{  mu: -1.000000e+00:  n = 76}{}\\
\Input{ \indent  4.606790e-02  0.000000e+00}{}\\
\Input{ \indent  4.606800e-02  3.837140e-02}{}\\
\Input{ \indent  9.213400e-02  4.393050e-02}{}\\
\Input{ \indent  1.842700e-01  4.977660e-02}{}\\
\Input{ \indent  2.764100e-01  4.806820e-02}{}\\
\Input{ \indent  3.685400e-01  4.385540e-02}{}\\
\Input{ \indent  6.449500e-01  2.695920e-02}{}\\
\Input{ \indent  7.370900e-01  2.255450e-02}{}\\
 \Input{ \indent }{ etc.}

\section{Input of $\pi(\Elab', \mulab \mid E)$ as a product, Eq.~(\ref{correlated})}
For tables of the energy-angle probability density $\pi(\Elab', \mulab \mid E)$
given as the product in Eq.~(\ref{correlated}), the identification line in
Section~\ref{model-info} is\\
      \Input{Process: Double differential EMuEpP data transfer matrix}{}\\
These data are always in the laboratory frame,\\
  \Input{Product Frame: lab}{}

The model-dependent portion of the input file in Section~\ref{model-info}
contains a section for the angular probability density $\pi_\mu(\mulab \mid E)$
and another for the conditional probability density $\pi_E(\Elab' \mid E, \mulab)$.

The section for angular probability density starts with the lines\\
  \Input{Angular data:}{$n = K$}\\
  \Input{Incident energy interpolation:}{probability interpolation flag}\\
  \Input{Outgoing cosine interpolation:}{list interpolation flag}\\
where $K$ is the number of incident energies~$E$.  The flag for
interpolation with respect to incident energy is one of those for
probability density tables in Section~\ref{interp-flags}, and that for 
the direction cosine $\mulab$ is one of those 
for simple lists.  There follows $K$ blocks of data,
one for each incident energy\\
    \Input{ Ein:}{$E$: \quad \texttt{n = $N$}}\\
indicating that data are given for $N$ pairs of values of $\mulab$
and $\pi_\mu(\mulab \mid E)$.

The section for conditional probability density of outgoing energy
$\pi_E(\Elab' \mid E, \mulab)$ gives the number~$K$
of incident energies along with the interpolation rules\\
  \Input{EMuEpPData:}{$n = K$}\\
  \Input{Incident energy interpolation:}{probability interpolation flag}\\
  \Input{Outgoing cosine interpolation:}{probability interpolation flag}\\
  \Input{Outgoing energy interpolation:}{list interpolation flag}\\
The flags for interpolation with respect to incident energy~$E$ and
direction cosine~$\mulab$ are those for probability density tables in
Section~\ref{interp-flags-probability}, and that for outgoing energy~$E$ is one of those
for simple lists.

For each incident energy~$E$ there is a data section of the form\\
    \Input{ Ein:}{$E$: \quad \texttt{n = $N$}}\\
indicating that data are given for $N$ values of $\mulab$.  The block of
data corresponding to a value of $\mulab$ is of the form\\
    \Input{ mu:}{$\mulab$: \quad \texttt{n = $J$}}\\
followed by $J$ pairs of values of outgoing energy $E$ and
probability density~$\pi_E(\Elab' \mid E, \mulab)$.

An example of this type of data with energy in MeV is given by\\
  \Input{Angular data: n = 13}{}\\
  \Input{Incident energy interpolation: lin-lin unitbase}{}\\
  \Input{Outgoing cosine interpolation: lin-lin}{}\\
  \Input{ Ein:   7.78148000e+00:  n = 5}{}\\
  \Input{ \indent   9.99788143e-01   5.88016882e+01}{}\\
  \Input{ \indent   9.99841107e-01   1.03998708e+03}{}\\
  \Input{ \indent   9.99894071e-01   1.70086214e+03}{}\\
  \Input{ \indent   9.99947036e-01   2.60922780e+03}{}\\
  \Input{ \indent   1.00000000e+00   2.70023193e+04}{}\\
  \Input{ $\cdots$}{}\\
   \Input{ Ein: 2.00000000e+02: n = 5}{}\\
 \Input{ \indent -1.00000000e+00  3.26136085e-01}{}\\
 \Input{ \indent -5.00000000e-01  3.82892835e-01}{}\\
  \Input{ \indent 0.00000000e+00  4.64096868e-01}{}\\
  \Input{ \indent 5.00000000e-01  5.89499334e-01}{}\\
  \Input{ \indent 1.00000000e+00  8.00885838e-01}{}\\
   \Input{EMuEpPData: n = 13}{}\\
  \Input{Incident energy interpolation: lin-lin unitbase}{}\\
  \Input{Outgoing cosine interpolation: lin-lin unitbase}{}\\
  \Input{Outgoing energy interpolation: lin-lin}{}\\
  \Input{ Ein:   7.78148000e+00:  n = 5}{}\\
  \Input{ mu:   9.99788143e-01:  n = 4}{}\\
  \Input{ \indent   2.35390141e-03  1.62759930e+05}{}\\
  \Input{ \indent   2.35697064e-03  1.62877493e+05}{}\\
  \Input{ \indent   2.35697074e-03  1.62877496e+05}{}\\
  \Input{ \indent   2.36004196e-03  1.62892892e+05}{}\\
  \Input{ mu:   9.99841107e-01:  n = 16}{}\\
  \Input{ \indent   2.30884914e-03  9.56657064e+03}{}\\
   \Input{ \indent  2.32094195e-03  9.99310479e+03}{}\\
   \Input{ \indent}{ etc.}\\
 \Input{ Ein:  2.00000000e+02: n = 5}{}\\
\Input{  mu: -1.00000000e+00: n = 501}{}\\
  \Input{ \indent 1.00000000e-18  5.38736174e-10}{}\\
  \Input{ \indent 1.00563208e-17  1.70842412e-09}{}\\
  \Input{ \indent 1.91126417e-17  2.35524717e-09}{}\\
  \Input{ \indent 2.81689625e-17  2.85931206e-09}{}\\
  \Input{ \indent 3.72252834e-17  3.28696542e-09}{}\\
  \Input{ $\cdots$}{}\\
  \Input{ mu: 1.00000000e+00: n = 993}{}\\
  \Input{ \indent 1.00000000e-18  2.19383712e-10}{}\\
  \Input{ \indent 7.55831305e-18  6.03138181e-10}{}\\
  \Input{ \indent }{ $\cdots$}\\
  \Input{ \indent 1.32751551e+01  1.88436981e-03}{}\\
  \Input{ \indent 1.38015128e+01  1.90038969e-03}{}
 
 